\vspace{-5pt} \hfill \newline
\textbf{New elements in this lab} \newline
\phantom{ } Since most of the basic elements have been discussed in the report of Lab 1, we will not introduce these elements again. We are going to discuss about the new elements we used in this lab. The two basic elements are 100$ \si{\kilo\Omega} $ resistors and 22$\si{\pico\farad}$ capacitor. The resistor are not like the one we discussed before, we will need to use a potentiometer with a maximum resistance of 100$ \si{\kilo\Omega} $ instead. We can change the resistance of a potentiometer rotating the knob (also remember to connect the middle pin to the circuit). The capacitor is almost same as the one we used to use, but with a different format of label. Also we will need to use Audio Jack, which contains three kinds of wires. The red wire is the high output source and the thick black one should be connected to the ground in the circuit. The other one can be left alone. We also used the op-amp LM348 in this lab. It is an IC-chip with 14 pins. the middle pin of the two sides is respectively $ \mathrm{V_{CC}} $ and $ \mathrm{V_EE} $. It contains 4 groups of pins. In each group, the pins at the ends is the output pin, the ones near the end is $ \mathrm{V_-} $ and the left one is $ \mathrm{V_+} $. The last new element we used is the speaker. It is like a audio player with two input terminals, $ \mathrm{V_+} $ and  $ \mathrm{V_-} $.
